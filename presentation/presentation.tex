% this is a beamer presentation file

\documentclass{beamer}
\usepackage{graphicx}
\usepackage{amsmath}
\usepackage{amssymb}
\usepackage{amsthm}
\usepackage{amsfonts}
\usepackage{amscd}
\usepackage{color}
\usepackage{enumerate}
\usepackage{hyperref}
\usepackage{url}
\usepackage{verbatim}
\usepackage{listings}
\usepackage{tikz}
\usepackage{pgfplots}

\title{PHYS 420: Ray Tracing Demonstration and Discussion}
\author{Ashtan Mistal}
\institute{University of British Columbia}
\date{\today}

\begin{document}

  \begin{frame}
    \titlepage
  \end{frame}

% Introduce the audience to what ray tracing is

  \begin{frame}
    \frametitle{What is Ray Tracing?}
    \begin{itemize}
      \item Ray Tracing is a method used to accurately model the propagation of light through a scene
%      \item It is used in many applications, such as computer graphics, astronomy, and optics
      \item Familiar applications of ray tracing include computer graphics, astronomy, and optics
    \end{itemize}
  \end{frame}

  \begin{frame}
    \frametitle{What are the advantages of ray tracing over conventional lighting models?}
    \begin{itemize}
      \item It can model the effects of light on objects more accurately than conventional lighting models
      \item It is more computationally intensive than conventional lighting models
    \end{itemize}

    In this presentation, we'll go over the details on why it models light more accurately, and how it is more computationally intensive.

    We'll also go over the details of how the model works, and how it can be implemented.
    % image for rasterized vs ray traced
    % https://appuals.com/ray-tracing-vs-rasterized-rendering-explained/
    \includegraphics[width=0.5\textwidth]{rasterized_vs_ray_traced.png}
  \end{frame}

  \begin{frame}
    \frametitle{Why is ray tracing more accurate than conventional lighting models?}

    Rasterized lighting is a computer graphics technique that is used to create shadows and other lighting effects in 3D scenes.
    The technique works by tracing rays of light from a light source through a 3D scene, and then calculating how these rays interact with the geometry and surfaces in the scene.
    This information is then used to generate a 2D image that shows the shadows and other lighting effects.



\end{document}
